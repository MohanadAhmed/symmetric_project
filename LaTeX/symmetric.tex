\documentclass[12pt,a4paper,reqno]{amsart}
\usepackage{hyperref}
\usepackage{cleveref}
\usepackage{amssymb}
\usepackage{verbatim}
\usepackage{amscd}
\usepackage{enumerate}
\usepackage{graphicx}
%\usepackage{showkeys}
\usepackage{siunitx}
\usepackage{tikz-cd}
\usepackage{stix}
\usepackage{bm}
\usepackage{mathtools}
\usepackage[tableposition=top]{caption}
\usepackage{booktabs,dcolumn}
\usepackage{tikz}
\usetikzlibrary{shapes.geometric, arrows}

\numberwithin{equation}{section}

\addtolength{\textwidth}{3 truecm}
\addtolength{\textheight}{1 truecm}
\setlength{\voffset}{-.6 truecm}
\setlength{\hoffset}{-1.3 truecm}

\theoremstyle{plain}

\newtheorem{theorem}{Theorem}[section]
\newtheorem{proposition}[theorem]{Proposition}
\newtheorem{lemma}[theorem]{Lemma}
\newtheorem{corollary}[theorem]{Corollary}
\newtheorem{conjecture}[theorem]{Conjecture}
\newtheorem{heuristic}[theorem]{Heuristic}
\newtheorem{principle}[theorem]{Principle}
\newtheorem{question}[theorem]{Question}
\newtheorem{claim}[theorem]{Claim}

\theoremstyle{definition}
\newtheorem{definition}[theorem]{Definition}
\newtheorem{remark}[theorem]{Remark}
\newtheorem{remarks}[theorem]{Remarks}
\newtheorem{example}[theorem]{Example}
\newtheorem{examples}[theorem]{Examples}

\renewcommand\P{\mathbf{P}}
\newcommand\E{\mathbf{E}}
\newcommand\Var{\mathbf{Var}}
\newcommand\R{\mathbb{R}}
\newcommand\Z{\mathbb{Z}}
\newcommand\N{\mathbb{N}}
\newcommand\C{\mathbb{C}}
\newcommand\Q{\mathbb{Q}}
\renewcommand\Re{{\operatorname{Re}}}
\renewcommand\Im{{\operatorname{Im}}}
\newcommand\eps{\varepsilon}

\parindent 0mm
\parskip   5mm


\begin{document}

\title{Proof of Maclaurin inequality}

\author{} %Terence Tao}
\address{}%UCLA Department of Mathematics, Los Angeles, CA 90095-1555.}
\email{}%tao@math.ucla.edu}

\subjclass[2020]{26D15}


\maketitle

%%%%%%%%%%%%%%%%%%%%%%%%%%%%%%%%%%%%%%%%%%%%%%%%%

\section{Introduction}

Given $n$ real numbers $y = (y_1,\dots,y_n) \in \R^n$ and $0 \leq k \leq n$, let $s_k(y)$ denote the elementary symmetric means
$$ s_k(y) \coloneqq \frac{1}{\binom{n}{k}} \sum_{1 \leq i_1 < \dots < i_k \leq n} y_{i_1} \dots y_{i_k}$$
(thus for instance $s_0(y)=1$).  

\begin{lemma}[Symmetries of attainable tuples]\label{syms}  Let $(s_0,\dots,s_n)$ be an attainable tuple.
    \begin{itemize}
        \item[(i)] (Scaling)  For any real $\lambda$, $(s_0, \lambda s_1, \dots, \lambda^n s_n)$ is attainable.
        \item[(ii)] (Reflection)  If $s_n \neq 0$, then $(1, s_{n-1}/s_n, \dots, s_0/s_n)$ is attainable. (In particular, if $|s_n|=1$, then $\pm (s_n,\dots,s_0)$ is attainable with $\pm$ the sign of $s_n$.)
        \item[(iii)] (Truncation) If $1 \leq \ell \leq n$, then $(s_0,\dots,s_\ell)$ is attainable.
    \end{itemize}
    \end{lemma}

    \begin{proof}  We can write $s_k = s_k(y_1,\dots,y_n)$ for some real $y_1,\dots,y_n$. The claims (i), (ii) are immediate from the homogeneity identity
        $$ s_k(\lambda y_1,\dots,\lambda y_n) = \lambda^k s_k(y_1,\dots,y_n)$$
        and the reflection identity
    $$ s_k(1/y_1,\dots,1/y_n) = s_{n-k}(y_1,\dots,y_n) / s_n(y_1,\dots,y_n)$$
    respectively for all $0 \leq k \leq n$ (note that the non-vanishing of $s_n(y_1,\dots,y_n)$ implies that all the $y_1,\dots,y_n$ are non-zero).  To prove (iii), observe from $n-\ell$ applications of Rolle's theorem that the degree $\ell$ polynomial
    $$ \frac{\ell!}{n!} \frac{d^{n-\ell}}{dx^{n-\ell}} \prod_{i=1}^n (z-y_i) = \sum_{k=0}^\ell (-1)^k \binom{\ell}{k} s_k(y_1,\dots,y_n) z^{\ell-k}$$
    is monic with all roots real, and hence the tuple $(s_0(y_1,\dots,y_n),\dots,s_\ell(y_1,\dots,y_n))$ is attainable.
\end{proof}



\end{document}
