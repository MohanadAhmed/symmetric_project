\documentclass[12pt,a4paper,reqno]{amsart}
\usepackage{hyperref}
\usepackage{cleveref}
\usepackage{amssymb}
\usepackage{verbatim}
\usepackage{amscd}
\usepackage{enumerate}
\usepackage{graphicx}
%\usepackage{showkeys}
\usepackage{siunitx}
\usepackage{tikz-cd}
\usepackage{stix}
\usepackage{bm}
\usepackage{mathtools}
\usepackage[tableposition=top]{caption}
\usepackage{booktabs,dcolumn}
\usepackage{tikz}
\usetikzlibrary{shapes.geometric, arrows}

\numberwithin{equation}{section}

\addtolength{\textwidth}{3 truecm}
\addtolength{\textheight}{1 truecm}
\setlength{\voffset}{-.6 truecm}
\setlength{\hoffset}{-1.3 truecm}

\theoremstyle{plain}

\newtheorem{theorem}{Theorem}[section]
\newtheorem{proposition}[theorem]{Proposition}
\newtheorem{lemma}[theorem]{Lemma}
\newtheorem{corollary}[theorem]{Corollary}
\newtheorem{conjecture}[theorem]{Conjecture}
\newtheorem{heuristic}[theorem]{Heuristic}
\newtheorem{principle}[theorem]{Principle}
\newtheorem{question}[theorem]{Question}
\newtheorem{claim}[theorem]{Claim}

\theoremstyle{definition}
\newtheorem{definition}[theorem]{Definition}
\newtheorem{remark}[theorem]{Remark}
\newtheorem{remarks}[theorem]{Remarks}
\newtheorem{example}[theorem]{Example}
\newtheorem{examples}[theorem]{Examples}

\renewcommand\P{\mathbf{P}}
\newcommand\E{\mathbf{E}}
\newcommand\Var{\mathbf{Var}}
\newcommand\R{\mathbb{R}}
\newcommand\Z{\mathbb{Z}}
\newcommand\N{\mathbb{N}}
\newcommand\C{\mathbb{C}}
\newcommand\Q{\mathbb{Q}}
\renewcommand\Re{{\operatorname{Re}}}
\renewcommand\Im{{\operatorname{Im}}}
\newcommand\eps{\varepsilon}

\parindent 0mm
\parskip   5mm


\begin{document}

\title{Proof of Maclaurin inequality}

\author{} %Terence Tao}
\address{}%UCLA Department of Mathematics, Los Angeles, CA 90095-1555.}
\email{}%tao@math.ucla.edu}

\subjclass[2020]{26D15}


\maketitle

%%%%%%%%%%%%%%%%%%%%%%%%%%%%%%%%%%%%%%%%%%%%%%%%%

\section{Introduction}

Given $n$ real numbers $y = (y_1,\dots,y_n) \in \R^n$ and $0 \leq k \leq n$, let $s_k(y)$ denote the elementary symmetric means
$$ s_k(y) \coloneqq \frac{1}{\binom{n}{k}} \sum_{1 \leq i_1 < \dots < i_k \leq n} y_{i_1} \dots y_{i_k}$$
(thus for instance $s_0(y)=1$).  
We call a $n+1$-tuple $(s_0,\dots,s_n)$ of real numbers \emph{attainable} if it is of the form $(s_0(y),\dots,s_n(y))$ for some $y$, or equivalently if the polynomial $\sum_{k=0}^n (-1)^k \binom{n}{k} s_k z^{n-k}$ is monic with all roots real.  We have previously proven



\begin{proposition}[Newton's inequality]\label{prop-newton} If $(s_0,\dots,s_n)$ is attainable, then $s_k s_{k+2} \leq s_{k+1}^2$ for all $0 \leq k \leq n$.
\end{proposition}

We now prove

\begin{theorem}[Maclaurin inequality]\label{thm-maclaurin}  Suppose that $(s_0,\dots,s_n)$ is an attainable tuple with all $s_i$ non-negative.  Then $s_\ell^{1/\ell} \leq s_k^{1/k}$ for all $1 \leq k \leq \ell \leq n$.
\end{theorem}

\begin{proof}  By induction on $\ell$ it suffices to verify the case $\ell=k+1$.

Suppose that $s_i=0$ for some $1 \leq i \leq k$.  From the Newton inequality $s_{i-1} s_{i+1} \leq s_i^2$ and non-negativity we conclude that $s_{i+1}=0$.  By induction we conclude that $s_{k+1}=0$ and the claim is true in this case.  Thus we may assume that $s_i \neq 0$ for all $0 \leq i \leq k+1$ (the case $i=0$ is easy since $s_0=1$).

Now write $d_i = s_i / s_{i-1}$ for $1 \leq i \leq k+1$, then the $d_i$ are positive.  The Newton inequality can then be rewritten as
$$ d_{i+1} \leq d_i$$
for all $1 \leq i \leq k$, while the Maclaurin inequality
$$ s_{k+1}^{\frac{1}{k+1}} \leq s_k^{\frac{1}{k}}$$
is equivalent to
$$ s_{k+1}^k \leq s_k^{k+1}$$
which expands to
$$ (d_1 \dots d_{k+1})^k \leq (d_1 \dots d_k)^{k+1}$$
which simplifies to
$$ d_{k+1}^k \leq d_1 \dots d_k$$
so it suffices to show that $d_{k+1} \leq d_i$ for all $1 \leq i \leq k$.  But this follows from the monotone decreasing nature of the $d_i$.
\end{proof}


\end{document}
